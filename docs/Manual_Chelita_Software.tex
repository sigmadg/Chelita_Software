\documentclass{article}
\usepackage{float}
\usepackage{arxiv}
\usepackage{graphicx}
\usepackage[utf8]{inputenc}
\usepackage[T1]{fontenc}
\usepackage[spanish]{babel}
\usepackage{hyperref}
\usepackage{url}
\usepackage{booktabs}
\usepackage{listings}
\usepackage{xcolor}
\usepackage{microtype}

\hypersetup{colorlinks=true, linkcolor=blue, urlcolor=blue}

\definecolor{codebg}{RGB}{248,249,250}
\lstset{
  inputencoding=utf8,
  backgroundcolor=\color{codebg},
  basicstyle=\ttfamily\small,
  breaklines=true,
  frame=single,
  framesep=3pt,
  numbers=left,
  numberstyle=\tiny,
  keywordstyle=\bfseries,
  showstringspaces=false,
}

% Título del resumen y keywords en español
\renewcommand{\keywordname}{{\bfseries \emph Palabras clave}}
\renewenvironment{abstract}{%
  \centerline{\large \bfseries \scshape Resumen}%
  \begin{quote}}{%
  \end{quote}}

\title{MANUAL DE CONFIGURACIÓN Y FUNCIONALIDAD \\ RETO CHELITA SOFTWARE -- FULLSTACK TEST}

\author{
  Ana Gabriela Ordoñez Güemes \\
  Chelita Software \\
  Documentación técnica \\
  \texttt{Reto Fullstack Test}
}

\begin{document}
\maketitle

\begin{abstract}
Este documento describe la configuración del proyecto (backend en FastAPI y frontend en React), los requisitos del sistema y la explicación de la funcionalidad de la aplicación: creación de documentos PDF a partir de un formulario y descarga mediante un código único de 10 caracteres.
\end{abstract}

\keywords{FastAPI \and React \and PDF \and API REST \and Vite \and ReportLab}

\section{INTRODUCCIÓN--}
\label{sec:intro}

La aplicación permite:
\begin{itemize}
  \item \textbf{Crear un documento PDF} rellenando un formulario con nombre, apellido, edad, teléfono y correo. El sistema genera un PDF con una plantilla fija y devuelve un \textbf{código único de 10 caracteres} que identifica ese documento.
  \item \textbf{Descargar un PDF ya creado} introduciendo el código de 10 caracteres en el frontend; la aplicación devuelve el documento en base64 y se descarga en el navegador.
\end{itemize}

La arquitectura consta de:
\begin{itemize}
  \item \textbf{Backend}: API REST en Python con FastAPI que genera los PDF (ReportLab), almacena los documentos en memoria y expone dos endpoints.
  \item \textbf{Frontend}: aplicación React (Vite) con formulario para crear documentos y sección para descargar por código.
\end{itemize}

\section{REQUISITOS DEL SISTEMA--}
\label{sec:requisitos}

\begin{itemize}
  \item \textbf{Python} 3.10 o superior (para el backend).
  \item \textbf{Node.js} 18 o superior y \textbf{npm} (para el frontend con Vite).
  \item Navegador web moderno (Chrome, Firefox, Edge, Safari).
\end{itemize}

\section{CONFIGURACIÓN DE LIBRERÍAS (DEPENDENCIAS)--}
\label{sec:libs}

Las dependencias del backend y del frontend (incluidas las de tests) se instalan por separado. Si la ruta del proyecto contiene espacios (por ejemplo \texttt{Chelita Software}), use comillas al escribir rutas completas o ejecute los comandos desde dentro de la carpeta correspondiente.

\subsection{Librerías del backend--}

El archivo \texttt{backend/requirements.txt} define todas las dependencias Python, tanto de ejecución como de tests:

\begin{itemize}
  \item \textbf{Ejecución}: \texttt{fastapi}, \texttt{uvicorn[standard]}, \texttt{reportlab}, \texttt{pydantic}.
  \item \textbf{Tests}: \texttt{pytest}, \texttt{pytest-asyncio}, \texttt{httpx} (cliente para TestClient).
\end{itemize}

\textbf{Instalación:} desde la carpeta del proyecto, active el entorno virtual del backend e instale:

\begin{lstlisting}[language=bash]
cd backend
source .venv/bin/activate    # Linux/macOS
# En Windows: .venv\Scripts\activate
pip install -r requirements.txt
\end{lstlisting}

Si no existe \texttt{.venv}, créelo antes con \texttt{python3 -m venv .venv}. No use una ruta con espacios sin comillas (ej.\ \texttt{"/ruta/Chelita Software/backend/.venv/bin/python"}).

\subsection{Librerías del frontend--}

El archivo \texttt{frontend/package.json} define dependencias de producción y de desarrollo (build y tests):

\begin{itemize}
  \item \textbf{Producción}: \texttt{react}, \texttt{react-dom}.
  \item \textbf{Desarrollo/build}: \texttt{vite}, \texttt{@vitejs/plugin-react}.
  \item \textbf{Tests unitarios}: \texttt{vitest}, \texttt{@testing-library/react}, \texttt{@testing-library/jest-dom}, \texttt{@testing-library/user-event}, \texttt{jsdom}.
  \item \textbf{Tests E2E}: \texttt{cypress}, \texttt{start-server-and-test}.
\end{itemize}

\textbf{Instalación:} desde la carpeta del proyecto:

\begin{lstlisting}[language=bash]
cd frontend
npm install
\end{lstlisting}

Con eso quedan instaladas todas las dependencias, incluidas las de tests. No es necesario instalar nada más para ejecutar \texttt{npm run test:run} ni \texttt{npm run e2e}.

\section{CONFIGURACIÓN--}
\label{sec:configuracion}

\subsection{Configuración del backend (FastAPI)--}

\begin{enumerate}
  \item Navegar a la carpeta del backend:
    \begin{lstlisting}[language=bash]
cd backend
    \end{lstlisting}
  \item Crear y activar un entorno virtual:
    \begin{lstlisting}[language=bash]
python3 -m venv .venv
source .venv/bin/activate    # Linux/macOS
# En Windows: .venv\Scripts\activate
    \end{lstlisting}
  \item Instalar dependencias:
    \begin{lstlisting}[language=bash]
pip install -r requirements.txt
    \end{lstlisting}
  \item Iniciar el servidor:
    \begin{lstlisting}[language=bash]
uvicorn app.main:app --reload --port 8000
    \end{lstlisting}
\end{enumerate}

El backend quedará disponible en \url{http://localhost:8000}. La documentación interactiva (Swagger) está en \url{http://localhost:8000/docs}.

\textbf{Parámetros opcionales:}
\begin{itemize}
  \item \texttt{--port}: puerto (por defecto 8000). Si se cambia, debe coincidir con la configuración del proxy del frontend o con la URL de la API en el frontend si se usa modo standalone.
  \item \texttt{--reload}: recarga automática al modificar código (solo desarrollo).
\end{itemize}

\subsection{Configuración del frontend (React + Vite)--}

\begin{enumerate}
  \item Navegar a la carpeta del frontend:
    \begin{lstlisting}[language=bash]
cd frontend
    \end{lstlisting}
  \item Instalar dependencias:
    \begin{lstlisting}[language=bash]
npm install
    \end{lstlisting}
  \item Iniciar el servidor de desarrollo:
    \begin{lstlisting}[language=bash]
npm run dev
    \end{lstlisting}
\end{enumerate}

La aplicación se abrirá en \url{http://localhost:5173}. El proxy de Vite redirige las peticiones a \texttt{/create} y \texttt{/document} al backend en \texttt{http://localhost:8000}, por lo que \textbf{el backend debe estar en ejecución} para que el frontend funcione correctamente.

\textbf{Archivo de configuración} \texttt{vite.config.js}:
\begin{lstlisting}
server: {
  port: 5173,
  proxy: {
    '/create': 'http://localhost:8000',
    '/document': 'http://localhost:8000',
  },
},
\end{lstlisting}

Si se cambia el puerto del backend, debe actualizarse la URL en \texttt{proxy}.

\subsection{Alternativa sin Node/npm (HTML standalone)--}

Si no se desea usar \texttt{npm install} ni \texttt{npm run dev}, puede utilizarse el archivo \texttt{frontend/index-standalone.html}, que carga React desde CDN y se conecta directamente al backend por URL.

\begin{enumerate}
  \item Asegurarse de que el backend esté corriendo en \texttt{http://localhost:8000}.
  \item Abrir en el navegador el archivo \texttt{frontend/index-standalone.html} (doble clic o ``Abrir con'' el navegador), o servir la carpeta \texttt{frontend} con un servidor HTTP y abrir ese archivo (por ejemplo, \texttt{python3 -m http.server 8080} y luego \url{http://localhost:8080/index-standalone.html}).
\end{enumerate}

En este modo, la URL de la API está fijada en el HTML (\texttt{http://localhost:8000}). El backend debe tener CORS configurado para el origen desde el que se sirve la página (por ejemplo \texttt{null} para \texttt{file://} o el origen del servidor usado).

\section{FUNCIONALIDAD--}
\label{sec:funcionalidad}

\subsection{Flujo general--}

\begin{enumerate}
  \item El usuario rellena el formulario ``Crear documento PDF'' con nombre, apellido, edad, teléfono y correo.
  \item Al enviar el formulario, el frontend llama al endpoint \texttt{POST /create} del backend con esos datos.
  \item El backend genera un PDF con la plantilla ``Chelita Software - Fullstack Test'' y los datos recibidos, lo guarda en memoria asociado a un código único de 10 caracteres alfanuméricos y devuelve ese código.
  \item El frontend muestra el código y permite copiarlo al portapapeles.
  \item Para descargar más tarde, el usuario introduce el código en la sección ``Descargar PDF por código'' y pulsa el botón. El frontend llama a \texttt{GET /document/\{code\}} y, con la respuesta en base64, provoca la descarga del PDF en el navegador.
\end{enumerate}

\subsection{API del backend--}

\subsubsection{POST /create--}

Crea un nuevo documento PDF a partir de los datos enviados en el cuerpo de la petición.

\textbf{Cuerpo de la petición} (JSON):
\begin{lstlisting}
{
  "nombre": "Alexis",
  "apellido": "Pina",
  "edad": "29",
  "telefono": "5581064181",
  "correo": "alexis.pina@chelita.com.mx"
}
\end{lstlisting}

\textbf{Respuesta exitosa} (200):
\begin{lstlisting}
{
  "success": true,
  "document_code": "EJ21243DIK"
}
\end{lstlisting}

\texttt{document\_code} es una cadena de exactamente 10 caracteres (letras mayúsculas y/o dígitos) que identifica de forma única el documento generado.

\subsubsection{GET /document/\{code\}--}

Devuelve el documento PDF asociado al \texttt{code} en formato base64.

\textbf{Parámetros:}
\begin{itemize}
  \item \texttt{code}: cadena de exactamente 10 caracteres (el \texttt{document\_code} devuelto por \texttt{POST /create}).
\end{itemize}

\textbf{Respuesta exitosa} (200):
\begin{lstlisting}
{
  "success": true,
  "document_b64": "jwiou2he1287ehiuhwadkjhei27h2i7he1i2e="
}
\end{lstlisting}

\texttt{document\_b64} es el contenido del PDF codificado en base64. El frontend decodifica este valor y genera un enlace de descarga (\texttt{data:application/pdf;base64,...}).

\textbf{Errores:}
\begin{itemize}
  \item \textbf{400}: el código no tiene 10 caracteres.
  \item \textbf{404}: no existe ningún documento almacenado con ese código.
\end{itemize}

\subsection{Contenido del PDF generado--}

El PDF generado tiene la siguiente estructura:
\begin{itemize}
  \item Título centrado: ``Chelita Software - Fullstack Test''.
  \item A continuación, los campos y sus valores:
  \begin{itemize}
    \item Nombre
    \item Apellido
    \item Edad
    \item Telefono
    \item Correo
  \end{itemize}
\end{itemize}

Se genera con la librería ReportLab en el backend. Los documentos se almacenan en memoria (diccionario código $\rightarrow$ bytes del PDF); si se reinicia el servidor, se pierden.

\subsection{Funcionalidad del frontend--}

\begin{itemize}
  \item \textbf{Formulario de creación}: campos obligatorios para nombre, apellido, edad, teléfono y correo; validación básica (required, tipo email). Al enviar, se muestra un indicador de carga y, al terminar, un mensaje de éxito o error y el código del documento.
  \item \textbf{Código del documento}: se muestra en un bloque destacado con botón ``Copiar'' para copiar el código al portapapeles.
  \item \textbf{Descarga por código}: campo de texto limitado a 10 caracteres (se fuerza mayúsculas), validación de longitud antes de enviar la petición y mensajes de error o éxito. Al recibir el PDF en base64, se dispara la descarga con nombre \texttt{documento-\{code\}.pdf}.
  \item \textbf{Accesibilidad}: etiquetas asociadas a los campos, mensajes de error con \texttt{role="alert"}, estados de carga con \texttt{aria-busy} y enfoque visible para teclado.
\end{itemize}

\section{EJECUCIÓN DE TESTS--}
\label{sec:tests}

A continuación se indica cómo ejecutar cada suite de tests. Las dependencias necesarias se instalan con los pasos de la sección \ref{sec:libs}.

\subsection{Resumen: comandos para ejecutar los tests--}

\begin{center}
\begin{tabular}{lll}
\toprule
\textbf{Suite} & \textbf{Dónde ejecutar} & \textbf{Comando} \\
\midrule
Tests backend (pytest) & \texttt{backend/} con venv activado & \texttt{pytest -v} \\
Tests frontend (Vitest) & \texttt{frontend/} & \texttt{npm run test:run} \\
Tests E2E (Cypress) & \texttt{frontend/} & \texttt{npm run e2e} \\
Test E2E API (script) & \texttt{backend/} con venv, servidor en marcha & \texttt{python ../scripts/e2e\_api\_test.py} \\
\bottomrule
\end{tabular}
\end{center}

\subsection{Tests del backend (pytest)--}

Desde la carpeta \texttt{backend}, con el entorno virtual activado (\texttt{source .venv/bin/activate} en Linux/macOS):

\begin{lstlisting}[language=bash]
pytest -v
\end{lstlisting}

Se ejecutan tests unitarios del servicio de generación de PDF (longitud y unicidad del código, generación de bytes PDF), de los endpoints (creación, obtención por código, validación 422, errores 400/404 y health) y un test de integración del flujo completo (crear documento y obtenerlo por código).

\subsection{Resultado de la ejecución (backend)--}

A continuación se muestra la salida típica de una ejecución exitosa de \texttt{pytest -v} en el backend:

\begin{lstlisting}[basicstyle=\ttfamily\footnotesize]
============================= test session starts =====
platform linux -- Python 3.13.3, pytest-8.3.4
rootdir: .../backend
configfile: pytest.ini
plugins: anyio, asyncio
collected 11 items

tests/test_api.py::test_create_document_... PASSED
tests/test_api.py::test_get_document_... PASSED
tests/test_api.py::test_get_document_invalid_404 PASSED
tests/test_api.py::test_get_document_bad_400 PASSED
tests/test_api.py::test_health PASSED
tests/test_api.py::test_create_validation_422 PASSED
tests/test_api.py::test_create_empty_field_422 PASSED
tests/test_integration_flow.py::test_full_flow... PASSED
tests/test_pdf_service.py::test_generate_code_length PASSED
tests/test_pdf_service.py::test_generate_code_... PASSED
tests/test_pdf_service.py::test_build_pdf_... PASSED

=================== 11 passed in 0.06s =================
\end{lstlisting}

\subsection{Tests del frontend (Vitest)--}
\label{sec:tests-frontend}

Desde la carpeta \texttt{frontend} (no es necesario tener el servidor \texttt{npm run dev} en marcha):

\begin{lstlisting}[language=bash]
cd frontend
npm run test:run
\end{lstlisting}

Se ejecutan tests con Vitest y React Testing Library: cliente API (\texttt{createDocument}, \texttt{getDocument}, \texttt{downloadPdfFromBase64}) y componente App (formularios, validación, flujo crear documento y descargar por código).

\subsection{Resultado de la ejecución (frontend)--}

Salida típica de \texttt{npm run test:run}:

\begin{lstlisting}[basicstyle=\ttfamily\footnotesize]
> vitest run

 RUN  v2.1.9

  src/api.test.js (6 tests) 9ms
  src/App.test.jsx (7 tests) 513ms

 Test Files  2 passed (2)
      Tests  13 passed (13)
   Duration  1.51s
\end{lstlisting}

\subsection{Tests E2E con Cypress--}

Desde la carpeta \texttt{frontend}. El comando \texttt{npm run e2e} arranca el servidor de desarrollo y luego ejecuta Cypress en modo headless; no hace falta tener el backend corriendo (la API se simula):

\begin{lstlisting}[language=bash]
cd frontend
npm run e2e
\end{lstlisting}

Para ejecutar solo los tests E2E contra la API real (backend y frontend en marcha en sus puertos): \texttt{npm run e2e:live}. Para abrir la interfaz gráfica de Cypress: \texttt{npm run cypress:open}.

\subsection{Test E2E de la API (script Python)--}

Con el backend corriendo en \url{http://localhost:8000}, desde la carpeta \texttt{backend} con el entorno virtual activado:

\begin{lstlisting}[language=bash]
cd backend
source .venv/bin/activate
python ../scripts/e2e_api_test.py
\end{lstlisting}

Salida esperada en caso de éxito:

\begin{lstlisting}[basicstyle=\ttfamily\footnotesize]
E2E API: comprobando flujo crear -> obtener...
  Health OK
  Crear documento OK (codigo: XXXXXXXXXX)
  Obtener documento OK (PDF valido)
E2E API: todos los pasos OK.
\end{lstlisting}

\section{RESUMEN DE ARCHIVOS RELEVANTES--}
\label{sec:archivos}

\begin{itemize}
  \item \texttt{backend/requirements.txt}: dependencias Python (FastAPI, ReportLab, etc.).
  \item \texttt{backend/app/main.py}: aplicación FastAPI y definición de endpoints.
  \item \texttt{backend/app/pdf\_service.py}: generación del código único y del PDF con ReportLab.
  \item \texttt{backend/app/store.py}: almacenamiento en memoria (código $\rightarrow$ PDF).
  \item \texttt{backend/app/models.py}: modelos Pydantic de request/response.
  \item \texttt{frontend/package.json}: dependencias y scripts npm (React, Vite).
  \item \texttt{frontend/vite.config.js}: configuración de Vite y proxy al backend.
  \item \texttt{frontend/src/App.jsx}: componente principal y flujo del formulario y descarga.
  \item \texttt{frontend/src/api.js}: llamadas a \texttt{/create} y \texttt{/document} y descarga desde base64.
  \item \texttt{frontend/src/components/}: componentes reutilizables (Card, FormField, Button, Message, CodeDisplay).
\end{itemize}

\vspace{1em}
\noindent\textit{Manual elaborado por Ana Gabriela Ordoñez Güemes.}

\end{document}
